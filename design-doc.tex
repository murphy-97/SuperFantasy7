\documentclass[titlepage]{article}
\usepackage{adjustbox}
\usepackage{graphicx}
\graphicspath{ {./images/} }


%% DOCUMENT OUTLINE
% Cover Page
% Overview
% What sets the project apart?
% Story and gameplay
% Schedule

\title{Shattered Reformation (Working)\\Super Fantasy 7}
\author{Savanna Middaugh, Hannah Murphy, Orion Nassaux, Jacob Santillanes}

\begin{document}
\begin{titlepage}
    \centering
    \vfill
    {\bfseries\Large
        "Shattered Reformation" Design Document\\
        Game by Super Fantasy 7\\
        \vskip2cm
        Savanna Middaugh, Hannah Murphy, Orion Nassaux, Jacob Santillanes\\
	v1.0.0
    }    
    \vfill
    \includegraphics[width=8cm]{./images/logo.png}
    \vfill
    \vfill
\end{titlepage}

\section{Overview}

\subsection*{Theme, Setting, Genre}
We will be designing a 3d metroidvania platformer that takes place in a (mostly)
 renaissance medieval, high fantasy, setting. 

\subsection*{Core Gameplay Mechanics (Brief)}
The core gameplay will consist of going through dungeons to get a special item 
and killing the boss at the end of the dungeon. The dungeons will have at least 
two rooms with enemies and one puzzle room. The enemies you kill will drop 
health so as to keep the player playing aggressively rather than defensively. As
 you kill enemies you will get xp and you can use this to upgrade your skill 
 with your items. 

\subsection*{Target Platform}
The game will be developed for PC using keyboard controls. Gamepad controllers 
will be implemented if time allows.

\subsection*{Project Scope}
The upper bound of the project's scope is:
\begin{itemize}
    \item Four unique bosses (two from a fantasy setting, two from a superhero setting)
    \item Four unique special items (two from a fantasy setting, two from a superhero setting)
    \item Four procedurally generated dungeons
\end{itemize}

The projects scope will begin much smaller and expand to the upper bound if time
and other resources allow. At a minimum, the projects scope will be:
\begin{itemize}
    \item Two unique bosses (one from a fantasy setting, one from a superhero setting)
    \item Two unique special items (one from a fantasy setting, one from a superhero setting)
    \item Two procedurally generated dungeons
\end{itemize}

Scope will expand first to three sets of bosses, unique items, and dungeons and
then expand to four if there is opportunity to expand the scope.

\subsection*{Game Time Scale}
There will be a working version of this game at the end of the 16 week time 
frame that is given.

\subsection*{Team Structure}
Jacob Santillanes (Ulteelectrom) is in the art section of the class and will be 
providing assets for the game. \\

Savanna Middaugh (savymidd) is in the art section of the class and will be 
providing assets for the game. \\

Hannah Murphy (murphy-97 on GitHub, HanJan on Discord) is in the programming
section of the class and will be implementing game systems in code and Unity.

\subsection*{Licenses/Hardware/Other Info}
The game will be made in Unity with assests created in Blender.  \\

Github repositiry: https://github.com/murphy-97/SuperFantasy7

\subsection*{Influences}
We took inspiration from several sources:

\begin{itemize}
    \item Metroidvania games: Our game will feature an open world of
    interconnected side-scolling platforming areas
    \item My Hero Academia, DC and Marvel: Half of the game premise draws from
    superhero media.
    \item Fantasy Genre: Half of the game premise draws from the fantasy genre.
    \item Skill Trees: Players will upgrade skills by spending points in a tree
    like in many other action-adventure games like Middle Earth: Shadow of
    Mordor and STAR WARS Jedi: Fallen Order.
    \item Legend of Zelda games: Dungeons feature puzzles, special items, and
    bosses.
    \item Rogue-like games: Dunegones are procedurally generated.
    \item Health Pickups: Players will recover health instantly from item
    pickups rather than maintaining an inventory, as in games like STAR WARS
    Battlefront (2004 and 2005 releases) and DOOM (2016). 
\end{itemize}

\subsection*{Elevator Pitch}
The game's working title is "Shattered Reformation" \\

Dark lords shattered the Planar Focus, creating chaos in the multiverse and
causing two worlds to merge. You have to defeat them and collect the Shards of
Alignment to reassemble the Planar Focus. Only then will the worlds be returned
to their natural orders.

\section{Unique Project Aspects}
% Premise: sci-fi plus fantasy
% Dungeons: balance between random generation and prefab set pieces
\begin{itemize}
	\item Features a mix of both classic fantasy and superhero themes.
	\item Uses randomised dungeons to deliver a unique experience in every game.
\end{itemize}

\section{Story \& Gameplay}
% Two worlds collided when the magic thingies were stolen
% The player must defeat fantasy monsters and super villains to bring alignment back to the multiverse
\subsection*{Story (Brief)}
The villains have merged two worlds together. One is a high fantasy setting and
one is a superhero setting. You need to separate the worlds and return
everything to normal. You play as either a hero or a knight. 

\subsection*{Gameplay (Brief)}
Basic platforming. The player can collect special items to interact with the
environment or find new ways of beating the enemies. The player will have a
basic attack, and a skill tree to upgrade their items and their basic attacks. 

\section{Schedule}
% Look at assignment due dates
% Start with two bosses and items
% Add a third boss and item if time
% Add a fourth boss and item if time

\subsection*{Phase 1: Conceptualization}

This phase is from 24 January 2020 through 10 February 2020. This phase will
flesh out early game design concepts, such as item, boss, and dungeons designs.
These assets will be designed mechanically and thematically, but no final coding
or art will be produced during this time.

\subsection*{Phase 2: Pre-Production}

This phase is from 10 February 2020 through 2 March 2020. This phase will
include development of early art assets and gray-box prototyping in Unity.
We will experiment with game systems and art individually.

\subsection*{Phase 3: Production}

This phase is from 2 March 2020 through 13 April 2020. This phase will see
finalization of assets, integration of assets in Unity, and full implementation
of all game mechanics. The game should be feature-complete by the end of this
phase.

\subsection*{Phase 4: Finalization}

This phase is from 13 April 2020 through 1 May 2020. No new features or assets
will be created during this time. Game features will be adjusted or fixed, and
that content which cannot be made playable will be cut.

% 03 Feb: Conceptualization 2
% 10 Feb: Conceptualization 3
% 24 Feb: Pre-Production 1
% 26 Feb: Midterm!
% 02 Mar: Pre-Production 2
% 23 Mar: Production 1
% 30 Mar: Production 2
% 06 Apr: Production 3
% 13 Apr: Production 4
% 27 Apr: QA/Polish 1
% 01 May: QA/Polish 2

\end{document}

    
